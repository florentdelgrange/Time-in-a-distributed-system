\documentclass[compress]{beamer}
\usepackage[utf8]{inputenc}
\usepackage[francais]{babel}
\usepackage[T1]{fontenc}
\usepackage{amssymb}
\usepackage{amsmath}
\usepackage{amsfonts}
\usepackage{hyperref}
\usepackage[]{algorithm2e}
\usepackage{amssymb}
\usepackage{verbatim}
\usepackage{listings}
\usepackage{color}
\usepackage{graphicx}
\usetheme[navigation]{UMONS}

\author{Clément Tamines, Florent Delgrange}
\title{Temps, Horloge et l'ordonnancement de évènements dans un système distribué}

\setbeamercovered{transparent} 
\setbeamertemplate{navigation symbols}{} 
\institute{UMONS\\Faculté des Sciences\\MA1 Sciences Informatiques\\[2ex]
  \includegraphics[height=4ex]{UMONS}\hspace{2em}%
  \raisebox{-1ex}{\includegraphics[height=6ex]{UMONS_FS}}}
\date{novembre 2016} 
\definecolor{darkgreen}{rgb}{0.0, 0.2, 0.13}
\subject{Réseaux II} 

\begin{document}

\begin{frame}
\titlepage
\end{frame}

\begin{frame}
\tableofcontents
\end{frame}

\section{Introduction}

\begin{frame}
\begin{itemize}
\item Comment classer chronologiquement les évènements dans un système distribué ?
\item Comment définir le fait qu'un évènement A s'est passé avant un autre évènement B ?
\end{itemize}
\end{frame}

\begin{frame}
\frametitle{Système distribué}
	\begin{definition}
		Un système distribué est une collection de processus distinct qui sont séparés dans l'espace et qui communiquent entre eux par 			messages.
	\end{definition}
\end{frame}

\end{document}