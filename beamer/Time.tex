\documentclass[compress]{beamer}
\usepackage[utf8]{inputenc}
\usepackage[francais]{babel}
\usepackage[T1]{fontenc}
\usepackage{amssymb}
\usepackage{amsmath}
\usepackage{amsfonts}
\usepackage{hyperref}
\usepackage[]{algorithm2e}
\usepackage{amssymb}
\usepackage{verbatim}
\usepackage{listings}
\usepackage{color}
\usepackage{graphicx}
\usetheme[navigation]{UMONS}

\author{Clément Tamines, Florent Delgrange}
\title[ ]{Time, Clocks, and the Ordering of Events in a Distributed System}

\setbeamercovered{transparent} 
\setbeamertemplate{navigation symbols}{} 
\institute{UMONS\\Faculté des Sciences\\MA1 Sciences Informatiques\\[2ex]
  \includegraphics[height=4ex]{UMONS}\hspace{2em}%
  \raisebox{-1ex}{\includegraphics[height=6ex]{UMONS_FS}}}
\date{novembre 2016} 
\definecolor{darkgreen}{rgb}{0.0, 0.2, 0.13}
\subject{Réseaux II} 
\begin{document}

\begin{frame}
\titlepage
\end{frame}

\begin{frame}
\tableofcontents
\end{frame}

\section{Introduction}

\begin{frame}
\frametitle{Problématiques}
Dans un système distribué, il n'est pas toujours possible de dire qu'un événement A s'est passé avant un événement B. \\ \bigskip
\begin{itemize}
\item Comment définir un ordre chronologique entre les évènements dans un système distribué ?
\item Comment définir qu'un évènement A précède un évènement B ?
\item Comment faire en sorte que cet ordre soit total ?
\item Comment assurer la fiabilité des mécanismes de synchronisation ?
\end{itemize}
\end{frame}

\begin{frame}
\frametitle{Système distribué}
	\begin{definition}
		Un système distribué consiste en un ensemble de processus séparés dans l'espace et qui communiquent entre eux via 			messages.
	\end{definition}
	\bigskip
	Exemples : 
	\bigskip
	\begin{itemize}
		\item Ordinateurs dans un réseau
		\item Plusieurs processus communiquant entre eux dans un ordinateur.
		\item Plusieurs threads d'un processeur
	\end{itemize}
\end{frame}

\begin{frame}
\frametitle{Temps physique}

Manière intuitive d'ordonner deux événements a et b :  \\ 
\begin{center}
 "a est arrivé avant b si a est arrivé plus tôt dans le temps que b"
\end{center}
Problèmes : \\
\bigskip
\begin{itemize}
\item Le temps doit être observable dans le système
\item Le système doit donc contenir des horloges 
\item Celles-ci ne sont pas parfaitement fiable et ne donnent pas assez précisément le temps physique
\end{itemize}
\bigskip
$\implies$ la relation \textit{est arrivé avant} doit s'exprimer sans horloge réelle.
\end{frame}

\section{Ordre partiel}

\begin{frame}
\frametitle{\'Evènements dans un processus}
\begin{columns}
    \begin{column}{0.2\textwidth}
    \includegraphics[scale=0.7]{clem1.png}
    \end{column}
  
    \begin{column}{0.8 \textwidth}
    \begin{itemize}
    \item Chaque processus est une suite d'événements
    \item La suite d'événements forme une séquence
    \item Dans la séquence, a se produit avant b ssi a est exécuté avant b
    \item La séquence est totalement ordonnée
    \item La réception et l'envoi de message sont des événements
    \end{itemize}
    \end{column}
\end{columns}
\end{frame}


\begin{frame}
\begin{definition}
La relation $\rightarrow$ sur un ensemble d'évènements d'un système satisfait
\begin{enumerate}
\item Si $a$ et $b$ sont deux évènements du même processus et a survient avant b, alors $a \rightarrow b$
\item Si $a$ correspond à l'envoi d'un message par un processus et b correspond à la réception de ce message par un autre processus, alors $a \rightarrow b$
\item $\rightarrow$ est transitif
\item Deux évènements $a, b$ sont concurrents ssi $a \not\rightarrow b$ et $b \not\rightarrow a$
\end{enumerate}
\end{definition}
\end{frame}

\begin{frame}
  \begin{columns}
    \begin{column}{.4\textwidth}
		\includegraphics[scale=0.19]{process2.png}
    \end{column}
	\begin{column}{0.6 \textwidth}
\begin{center}
$p_1 \rightarrow r_4$  par transitivité avec \\
\bigskip
	$p_1 \rightarrow q_2$\\$q_2 \rightarrow q_3$ \\ $q_3 \rightarrow q_4$ \\ $q_4 \rightarrow r_3$ \\ $r_3 \rightarrow r_4$\\
\end{center}
	\end{column}
	\end{columns}
	\bigskip	\textbf{{\color{red}Problème : }}$q_3$ est concurrent avec $p3$ : on ne peut pas déterminer si $q_3$ s'est passé avant $p_3$ ou vice versa.
\end{frame}

\section{Horloges Logiques}

\begin{frame}
\frametitle{Horloges logiques}
Une horloge est une façon d'assigner un nombre à un évènement où le nombre correspond au temps où l'évènement s'est produit.\\\bigskip
$\implies$ Chaque processus $p_i$ possède une horloge $C_i$.\\
$\implies$ Une horloge $C_i$ est une fonction qui associe un nombre $C_i$<$a$>  à chaque événement $a$ dans $P_i$\\
$\implies$ Le système entier est représenté par la fonction $C$.
\begin{block}{Propriétés}
\begin{itemize}
\item Soit b, un évènement, $C$<$b$>$ = C_j$<$b$> ssi $b$ est un évènement du processus $P_j$.\\
\item $\forall i$, $C_i$ peut être simplement implémenté à l'aide de compteurs sans mécanisme lié au temps.
\end{itemize} 
\end{block}
\end{frame}

\begin{frame}
\frametitle{Conditions}
\begin{block}{Condition faible}
$a \rightarrow b \implies C$<$a$> $ < C$<$b$>
\end{block}
\bigskip
Soient $a, b$ deux événements concurrents. \\On a $a \not\rightarrow b$  et $a \not\rightarrow b$ donc\\
\begin{center}
$C$<$a$>  $\geq$ $C$<$b$> $\land$ $C$<$a$>  $\leq$ $C$<$b$>  $\implies$ $C$<$a$> = $C$<$b$>
\end{center}
Les événements $a$ et $b$ doivent être simultanés !
\end{frame}
\begin{frame}

\begin{columns}
    \begin{column}{.35\textwidth}
		\includegraphics[scale=0.15]{process2.png}
    \end{column}
	\begin{column}{0.65 \textwidth}
\textbf{{\color{red}Problème : }}$p_2$ et $p_3$ sont concurrents de $q_3$. Donc $C$<$p_2$> = $C$<$p_3$> = $C$<$q_3$>. \\
Mais $p_2 \rightarrow p_3$ donc $C$<$p_2$> $<$ $C$<$p_3$>\\ \bigskip
$\implies$ Besoin de conditions plus fortes
	\end{column}
	\end{columns}
\begin{block}{Conditions fortes}
\begin{enumerate}
\item Si $a$ et $b$ sont des évènements de $P_i$ et que $a$ vient avant $b$, alors $C_i$<$a$> $<$ $C_i$<$b$>
\item Si $a$ est l'envoi d'un message par un processus $P_i$ et que $b$ est la réception de ce message par le processus $j$, alors $C_i$<$a$> $< C_j$<$b$>
\end{enumerate}
\end{block}
\end{frame}

%\begin{frame}
%\begin{figure}
%		\includegraphics[scale=0.17]{process3.png}

%\end{figure}
%\textbf{Exemples : }\\
%		$C$<$p_1$> $< C$<$p_3$>\\
%		$C$<$q_5$> $< C$<$p_4$>
%\end{frame}

\begin{frame}
\frametitle{Implémentation}
Une implémentation possible est décrite par les  
\textbf{\color{cyan}règles d'implémentations }suivantes : \bigskip
\begin{block}{RI 1} 
Chaque processus $P_i$ incrémente le compteur $C_i$ entre 2 évènements successifs.\\
Soient $a$ et $b$, deux évènements successifs de $P_i$. \\
Alors après $a$, $C_i \leftarrow C_i + 1$.
\end{block}
\end{frame}

\begin{frame}
	\frametitle{Implémentation}
	\begin{block}{RI 2}
	\begin{enumerate}[(a)]
		\item Si un évènement $a$ est l'envoi d'un message $m$ par le processus $P_i$, alors le message $m$ contient un Timestamp $T_m \leftarrow C_i$<$a$>
		\item Lorsqu'il reçoit un message $m$, le processus $P_j$ assigne la valeur $\mu$ à son compteur $C_j$ tel que $\mu > T_m$ et $\mu \geq$ valeur de $C_j$ avant la réception de $m$
	\end{enumerate}
	\end{block}
	\bigskip
	L'utilisation d'un Timestamp permet au processus qui reçoit un message de mettre à jour son compteur pour que que celui-ci respecte bien RI 1 (ordre dans ce même processus) et RI 2 (ordre entre les processus).
\end{frame}

\section{\'Evènements totalement ordonnés}
\begin{frame}
\frametitle{Ordre total}
On souhaite avoir un système d'horloge qui satisfait les conditions d'horloge précédemment énoncées et qui met en place un \textit{\textbf{ordre total}}.\\ \bigskip
\begin{block}{Ordre Total}
Une \textbf{relation binaire} $\preceq$ sur un ensemble E est un ordre total si $\forall x, y, z \in E$, \\
\begin{itemize}
\item \textbf{Réflexivité :} $x \preceq x$
\item \textbf{Antisymétrie :} $x \preceq y$ et $y \preceq x \implies x = y$
\item \textbf{Transitivité :} $x \preceq y$ et $y \preceq z \implies x \preceq z$
\end{itemize}
\end{block}
\end{frame}

\begin{frame}
\frametitle{Ordre Total}
Soit $\prec$ un ordre total arbitraire (utilisé en cas d'égalité)\\ \bigskip
On définit la relation $\Rightarrow$ comme suit :\\
\begin{center}
Si $a \in P_i$ et $b \in P_j$, alors \\
\[
	a \Rightarrow b \ \ \text{ssi}
	\begin{cases}
		C_i \text{<}a\text{>} < C_j\text{<}b\text{>} \\
		\text{ou}\\
		C_i\text{<}a\text{>} = C_j\text{<}b\text{>} \text{ et } P_i \prec P_j
	\end{cases}
\]
\end{center}
La relation $\Rightarrow$ est un ordre total (elle étend la relation  $\rightarrow$ )\\
\bigskip
Il est important de pouvoir ordonner totalement les événements dans un système distribué pour en assurer le bon fonctionnement.
\end{frame}

\begin{frame}
\frametitle{Application : partage de ressource}
\begin{itemize}
\item Considérons un système composé d'un ensemble fini de processus qui partagent une ressource unique.
\item On veut que les processus se synchronisent et évitent les conflits.
\end{itemize}
\begin{block}{Propriétés}
\begin{enumerate}
\item Le processus qui utilise la ressource doit la libérer avant que celle-ci puisse être utilisée par un autre processus
\item Plusieurs demandes différentes pour la ressource doivent être accordées dans l'ordre dans lesquelles elles ont été faites
\item Si chaque processus qui détient la ressource à un moment donné la libère, alors chaque requête sera accordée à un moment donné
\end{enumerate}
\end{block}
\end{frame}

\begin{frame}
\frametitle{Horloge centrale}
\textbf{{\color{red}Problème : }} Un système d'horloge centrale qui alloue la ressource en fonction des requêtes qu'elle reçoit ne respecte pas les conditions.\\\bigskip

\begin{columns}
    \begin{column}{.4\textwidth}
		\includegraphics[scale=0.55]{process4.png}
    \end{column}
	\begin{column}{0.6 \textwidth}
	\begin{itemize}
	\item 	$p_2$ envoie sa demande de ressource à $p_0$\\
	\item  	$p_2$ envoie ensuite un message à $p_1$\\
	\item 	suite à la réception du message, $p_1$ fait aussi une demande de la ressource à $p_0$\\
	\item Il est possible que la demande de $p_1$ arrive à $p_0$ avant celle de $p_2$.
	\end{itemize}
 \bigskip Les requêtes ne seront pas accordées selon l'ordre dans lequel elles ont été faites !
	\end{column}
	\end{columns}
\end{frame}

\begin{frame}
Nous allons introduire un système d'horloges avec les règles faibles et fortes énoncées précédemment pour avoir un ordre total $\Rightarrow$ sur les évènements.\\ \bigskip
Afin de simplifier le problème nous ferons les suppositions suivantes : \\ \bigskip
\begin{enumerate}
\item On suppose que pour tout processus $P_i$ et $P_j$, les messages envoyés de $P_i$ à $P_j$ sont reçus dans le même ordre qu'ils sont envoyés. 
\item On suppose que chaque message envoyé finit toujours par être reçu.
\item On suppose qu'un processus peut envoyer des messages vers tout autre processus.
\end{enumerate}
Notons que les deux premières suppositions peuvent être évitées en utilisant des numéros de messages et un protocole d'accusé de réception.
\end{frame}

\begin{frame}
\frametitle{Algorithme}
\begin{itemize}
\item Tout processus maintient une file de requête (file de priorité) invisible à tout autre processus.
\item Une demande de ressource par le processus $p_i$ est de la forme $T_m:P_i$ où $T_m$ est le timestamp du message de requête.
\item Initialisation de la file : {$T_0:P_0$} avec $\forall i,  T_0 \leq C_{i_0}$ et $P_0$, le processus auquel la ressource est accordée à l'initialisation.
\end{itemize}
	\begin{columns}
    	\begin{column}{.5\textwidth}
			Pour demander la ressource, $P_i$ envoie un message de demande $T_m : P_i$ à tous les autres processus et place ce message sur sa file de requête
		\end{column}
		\begin{column}{.5\textwidth}
		\begin{center}
		\includegraphics[scale=0.6]{process8.png}
		\end{center}
		
		\end{column}
	\end{columns}
%\includegraphics[scale=0.1]{process9.png}\includegraphics[scale=0.1]{process10.png}\includegraphics[scale=0.1]{process11.png}
\end{frame}

\begin{frame}
\frametitle{Algorithme}
	\begin{columns}
    	\begin{column}{.6\textwidth}
			Lorsque le processus $P_j$ reçoit le message $T_m : P_i$, il le place dans sa file de requête et envoie un accusé de réception à $P_i$
		\end{column}
		\begin{column}{.4\textwidth}
		\begin{center}
				\includegraphics[scale=0.5]{process9.png}
		\end{center}
		\end{column}
	\end{columns}
	
		\begin{columns}
    	\begin{column}{.6\textwidth}
			Pour libérer la ressource, le processus $P_i$ supprime tout $T_m : P_i$ de sa file de requête et envoie un message $P_i$ qui notifie la libération de la ressource à chaque autre processus.
		\end{column}
		\begin{column}{.4\textwidth}
		\begin{center}
				\includegraphics[scale=0.5]{process10.png}
		\end{center}
		\end{column}
	\end{columns}
\end{frame}

\begin{frame}
\frametitle{Algorithme}
\begin{columns}
    	\begin{column}{.6\textwidth}
			Lorsque le processus $P_j$ reçoit un message de libération de ressource $P_i$, il supprime chaque requête $T_m : P_i$ de sa file de requête.
		\end{column}
		\begin{column}{.4\textwidth}
		\begin{center}
		\includegraphics[scale=0.5]{process11.png}
		\end{center}
	
		\end{column}
	\end{columns}
	\bigskip
	Finalement, la ressource est accordée au processus $P_i$ lorsque \bigskip
	\begin{itemize}
	\item Il y a un message contenant $T_m : P_i$ dans sa file de requête qui est ordonnée avant chaque autre requête dans sa file par la relation $\Rightarrow$.
	\item $P_i$ a reçu un message de tous les autres processus daté après $T_m$.
	\end{itemize}
	\bigskip
	{\color{cyan} Chaque processus} suit ces règles indépendamment donc on a pas de processus de synchronisation central.
\end{frame}

\begin{frame}
\frametitle{Machine à états}
\begin{block}{Machine à états $SM(S, C, \delta)$}
\begin{itemize}
\item $S$ ensemble des états possibles. \\Un état $s$ correspond à une file de requêtes où la requête à la tête de la file est actuellement accordée.
\item $C$ ensemble des commandes possibles. \\Correspond aux commandes $P_i $ de demande ou de libération de ressource.
\item $\delta : C \times S \rightarrow S$ une fonction de transition telle que $\delta(c, s) = s'$ indique qu'exécuter l'action $c$ quand la machine est en l'état $s$ amène celle-ci en l'état $s'$.
\end{itemize}
\end{block}
\end{frame}

\begin{frame}
\frametitle{Simple Exemple}
\begin{figure}
	\includegraphics[scale=0.2]{state_machine.png}
\end{figure}
Chaque processus exécute indépendamment un tel automate. \\
En pratique, l'automate est plus complexe incluant des commandes avec des Timestamp $T_m$ et les commandes provenant des autres processus (ACK, request, release,...).
\end{frame}

\begin{frame}
\frametitle{Points faibles}
\begin{itemize}
\item L'algorithme nécessite la participation active de tous les autres processus
\item Un processus \textbf{doit} connaître toutes les commandes issues des autres processus. 
\item L'échec d'un processus peut dérégler tout le système.
\item Sans \textbf{\color{cyan} temps physique}, il n'y a pas moyen de distinguer l'échec d'un processus. La seule manière de détecter cet échec est un timeout, qui nécessite l'utilisation de temps physique.
\end{itemize}
Cependant, le fait que l'algorithme classe les requêtes selon l'ordre total $\Rightarrow$ permet de détecter les comportements anormaux.
\end{frame}

% /!\ J'ai pas compris la suite Anomalous Behavior /!\

%\section{Horloges Physiques}
%\begin{frame}
%\frametitle{Horloges Physiques : cas continu}
%Soit $C_i(t)$, la lecture de l'horloge $C_i$ au temps physique $t$.
%\begin{itemize}
%\item On suppose que les horloges tournent en continu.
%\item On passe des cas discrets au cas continus.
%\item On suppose que $C_i(t)$ est \textit{dérivable},\\
%	sauf pour les sauts discontinus où l'horloge est réinitialisée.
%\item $\frac{\partial C_i(t)}{\partial t}$ représente la vitesse à laquelle l'horloge $C_i(t)$ fonctionne au temps $t$.
%\item Ce taux devrait être $\approx 1$.
%\end{itemize}
%\end{frame}
%
%\begin{frame}
%\frametitle{Propriétés}
%\begin{block}{PC 1}
%$\exists \kappa \ll 1$ tel que $\forall i, |\frac{\partial C_i(t)}{\partial t} - 1| < \kappa$
%\end{block}
%exemple : \textit{horloges à cristal $\kappa \leq 10^{-6}$}
%Cette condition n'est pas assez forte : les horloges doivent être synchronisées entre elles c.à.d.
%\[\forall i, j, t \ \ C_i(t) \approx C_j(t)\]
%\begin{block}{PC 2}
%$\exists \epsilon \ll 1, \forall i, j \ \ |C_i(t) - C_j(t)| < \epsilon$
%\end{block}
%$\kappa$ et $\epsilon$ doivent assurer d'éviter les anomalies.\\
%\end{frame}
%
%\begin{frame}
%\frametitle{\'Eviter les comportements anormaux}
%Soient $\mu \in \mathbb{R}^{> 0}$ , $a, b$ des évènements de deux processus distincts et le temps physique $t$.\\
%\textbf{Si $a$ se produit en $t$ et que $b$ satisfait {\color{cyan}$a \rightarrow b$}, alors $b$ se produit après {\color{cyan}$t + \mu$}}, $\mu$ est le plus petit temps de transmission entre deux processus.\\
%exemple : Soient $sp$, le plus court chemin entre les processus et $c$, la vitesse de la lumière. On peut prendre $\mu = \frac{sp}{c}$
%\begin{block}{Propriétés}
%On suppose également que quand l'horloge est réinitialisée, elle est toujours remise en avant et jamais en arrière. Alors, pour éviter les comportements anormaux :
%\[ C_i(t+\mu) - C_i(t) > (1 - \kappa)\mu\]
%\[ C_i(t+\mu) - C_j(t) > 0 \text{ avec } \frac{\epsilon}{1-\kappa} \leq \mu \text{ (par PC 2)} \]
%\end{block}
%\end{frame}
%
%\begin{frame}
%\frametitle{Délai d'envoi}
%Soit $m$, un message envoyé au temps physique $t$ et reçu au temps $t'$.\\
%{\color{cyan}$v_m = t' - t$} est le \textbf{\color{cyan}délai total} du message $m$.\\
%Ce délai est non connu du processus qui reçoit $m$. On suppose par contre que tout processus recevant $m$ connait le\\ \textbf{\color{cyan} délai minimal de $m$ : } $0 \leq \mu_m \leq v_m$. \\
%\textbf{\color{cyan} délai imprévisible de $m$ : }$\xi_m = v_m - \mu_m$
%\end{frame}
%
%\begin{frame}
%\frametitle{Implémentation}
%\begin{block}{IR 1'}
%$\forall i$, si $P_i$ ne reçoit pas un message au temps physique $t$, alors $C_i$ est dérivable en t et $\partial_t C_i (t) > 0$
%\end{block}
%
%\begin{block}{IR 2'}
%\begin{enumerate}[(a)]
%\item Si $P_i$ envoie un message $m$ au temps physique $t$, alors $m$ contient un timestamp $T_m \leftarrow C_i(t)$.
%\item Lorsqu'on reçoit un message $m$ au temps $t'$, le processus $P_j$ met à jour $C_j(t')$ : 
%\[C_j(t') = \max (C_j(t'-0), T_m + \mu_m)\]\[\text{où } C_j(t'-0) = \lim_{\delta \rightarrow 0} C_j(t' - |\delta|)\]
%\end{enumerate}
%\end{block}
%
%\end{frame}
%% Préviens si tu captes la fin, j'y comprends pas grand chose. Je tente quand même ma chance .
%
%\begin{frame}
%\frametitle{PC2 satisfait ?}
%On veut à présent montrer que \textbf{\color{cyan}PC2 est satisfait}.\\
%On suppose que le système de processus est modélisé par un graphe :
%\begin{figure}
%	\includegraphics[scale=0.2]{process_graph.png}
%\end{figure}
%\begin{itemize}
%\item Un message est envoyé sur cet arc toutes les $\tau$ secondes si $\forall t$, $P_i$ envoie au moins un message vers $P_j$ entre $t$ et $t+\tau$.
%\item \textbf{\color{cyan}Diamètre $d$ du graphe} : le plus petit $d$ tel que pour toute paire $(P_i, P_j), \exists$ un chemin de $P_j$ à $P_k$ d'au moins $d$ arcs.
%\end{itemize}
%\end{frame}
%
%\begin{frame}
%\frametitle{PC2 satisfait ?}
%\begin{itemize}
%\item On suppose donc que le graphe est \textbf{fortement connexe} et de \textbf{diamètre} $d$.
%\item Le graphe obéi aux règle d'implémentation IR 1' et IR2'.
%\item $\forall$ message $m$, $\mu_m \leq \mu$ pour une constante $\mu$ et $\forall t \geq t_0$.
%	\begin{enumerate}[(a)]
%	\item PC1 est respecté.
%	\item $\exists \tau, \xi$ tels que toutes les $\tau$ secondes, un message avec un délai imprévisible plus petit que $\xi$ est envoyé sur chaque arc.
%	\end{enumerate}
%	Alors, PC2 est satisfait avec 
%	\[\epsilon \approx d(2 \kappa \tau + \xi) \forall t \geq t_0+\tau d\]
%	où l'approximation est $\mu + \xi \ll \tau$.
%\end{itemize}
%\end{frame}

\end{document}